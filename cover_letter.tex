% Letter template based on StackExchange
% Werner: https://tex.stackexchange.com/a/583808
% Accessed 2022-09-12

\documentclass{article}

\usepackage[margin=1.25in]{geometry}
\usepackage{fancyhdr,graphicx}

\fancypagestyle{firstpage}{
    \fancyhf{} % Clear header/footer
    \renewcommand{\headrulewidth}{0pt}%
    \renewcommand{\footrulewidth}{0pt}%
}

\AtBeginDocument{\thispagestyle{firstpage}}

\setlength{\parindent}{0pt}
\setlength{\parskip}{1ex}

\begin{document}

\hfill
\begin{tabular}{ l @{} }
    Colin Sullender\\
    The University of Texas at Austin\\
    Department of Biomedical Engineering\\
    107 W. Dean Keeton Street Stop C0800\\
    Austin, TX, 78712, USA
\end{tabular}

\bigskip\bigskip

\begin{tabular}{ @{} l }
    \today \\[12pt] % Date
\end{tabular}
  
\bigskip

Dear Dr. Pogue,

\bigskip

We wish to submit a new manuscript entitled ''Using pressure-driven flow systems to evaluate laser speckle contrast imaging'' for consideration by \emph{Journal of Biomedical Optics}. We confirm that this work is original and has not been published elsewhere nor is it currently under consideration in another journal.

In this paper, we report on the characterization of laser speckle contrast imaging (LSCI) systems when using microfluidic flow phantoms. We evaluated the flow generation performance of a widely-used syringe pump controller compared to a pressure-regulated flow control system. The pressure-regulated system greatly outperformed the syringe pump across a range of flow speeds both in terms of stability and accuracy. The pressure-regulated flow system was then used to assess the performance of both single-exposure LSCI and multi-exposure speckle imaging (MESI) during stepped flow profile experiments. Consistent with prior work, MESI outperformed single-exposure LSCI at all tested flow speeds and more closely mirrored the programmed flow.

This work is significant because it demonstrates the importance of selecting a reliable flow control system when conducting microfluidic characterization studies. The use of syringe pumps introduces additional uncertainty to the resulting measurements that could mask the true sensitivity of an LSCI instrument. Our results suggest that pressure-regulated flow systems should be used instead of syringe pumps when assessing the performance of flow measurement techniques. While we only evaluated LSCI/MESI, these findings are likely relevant for the broader dynamic light scattering community.

Please address all correspondence to Dr. Andrew Dunn at \underline{adunn@utexas.edu}.

Thank you for your consideration.

\bigskip

Sincerely,

\vspace{50pt}

Colin Sullender

\end{document}
